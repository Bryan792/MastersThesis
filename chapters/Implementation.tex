\chapter{Implementation}

There exists three main steps in my implementation, the construction of the suffix array, the calculation of ANSV for every index, and finally the generation of the LZ factorization.

\section{SA}


\begin{figure}
\begin{algorithmic}[1]
\Procedure{computeSA}{$s,sa,n$}
\State initMod12();\Comment{Kernel to set flags at 2/3. DeviceSelect to get s12,sa12}
\State radixSort($s12$);\Comment{DeviceRadixSort}
\State radixSort($s12$); 
\State radixSort($s12$); 
\State lexicRankOfTriplets();\Comment{Custom kernel to check unique. Inclusive Sum to count. Custom kernel to get s12.}
\If{!allUniqueRanks}
\State computeSA($s12,sa12$);\Comment{Recursion}
\State storeUniqueRanks();\Comment{Kernel}
\Else
\State computeSAFromUniqueRank();\Comment{Kernel}
\EndIf
\State radixSort($s0$); 
\State mergeSort($s0,s12$);\Comment{Merge Path + Merge Sort} 
\EndProcedure
\end{algorithmic}
\caption{Suffix Array Construction Pseudocode from \cite{ }. Comments add details from our implementation.}
\label{euclid}
\end{figure}

\begin{figure}
\begin{algorithmic}[1]
\Procedure{ANSV}{}
\For{each level of MinTree}\Comment{Bottom Up Construction}
\State MinTree($level$);\Comment{Build level by calculating minima of children}
\EndFor
\State ANSVKernel($mintree,chunkSize$)
\EndProcedure
\end{algorithmic}
\begin{algorithmic}[1]
\Procedure{ANSVKernel}{}
\State $chunk \gets threadID * chunkSize$\Comment{Each thread gets a unique chunk}
\State ANSVLinear($chunk$)
\If($chunk$ detects no PSV/NSV)\Comment{ANSVLinear may be wrong}
\State checkMinTree($mintree$)\Comment{Manually check MinTree}
\EndIf
\EndProcedure
\end{algorithmic}
\caption{ANSV Generation}
\end{figure}

\begin{table}[h]
\begin{tabular}{llll}
sa & suf-lex        & nsv-lex & psv-lex \\
8  & aaabab         & 3       & -1      \\
9  & aabab          & 3       & 8       \\
3  & aabbbaaabab    & 0       & -1      \\
12 & ab             & 10      & 3       \\
10 & abab           & 0       & 3       \\
0  & abbaabbbaaabab & -1      & -1      \\
4  & abbbaaabab     & 2       & 0       \\
13 & b              & 7       & 4       \\
7  & baaabab        & 2       & 4       \\
2  & baabbbaaabab   & 1       & 0       \\
11 & bab            & 6       & 2       \\
6  & bbaaabab       & 1       & 2       \\
1  & bbaabbbaaabab  & -1      & 0       \\
5  & bbbaaabab      & -1      & 1      
\end{tabular}
\end{table}

\begin{table}[h]
\begin{tabular}{lllllllll}
i  & S{[}i{]} & nsv-lex & nsv-match & psv-lex & psv-match & LPF{[}i{]} & PrevOcc{[}i{]} & LZ \\
0  & a        & -1      & -         & -1      & 0         & -1         & 0              &    \\
1  & b        & -1      & -         & 0       & -         & 0          & -1             & 1  \\
2  & b        & 1       & b         & 0       & -         & 1          & 1              & 2  \\
3  & a        & 0       & a         & -1      & -         & 1          & 0              & 3  \\
4  & a        & 2       & -         & 0       & abb       & 3          & 0              & 4  \\
5  & b        & -1      & -         & 1       & bb        & 2          & 1              & -  \\
6  & b        & 1       & bbaa      & 2       & bb        & 4          & 1              & -  \\
7  & b        & 2       & baa       & 4       & -         & 3          & 2              & 7  \\
8  & a        & 3       & aa        & -1      & -         & 2          & 3              & -  \\
9  & a        & 3       & aab       & 8       & aa        & 3          & 3              & -  \\
10 & a        & 0       & ab        & 3       & a         & 2          & 0              & 10 \\
11 & b        & 6       & b         & 2       & ba        & 2          & 2              & -  \\
12 & a        & 10      & ab        & 3       & a         & 2          & 10             & 12 \\
13 & b        & 7       & b         & 4       & -         & 1          & 7              & - 
\end{tabular}
\end{table}

\begin{table}[h]
\begin{tabular}{llllll}
i  & S{[}i{]} & NSV{[}i{]} & PSV{[}i{]} & LPF{[}i{]} & LZ \\
0  & a        & -1         & -1         & 0          & 0  \\
1  & b        & -1         & 0          & 0          & 1  \\
2  & b        & 1          & 0          & 1          & 2  \\
3  & a        & 0          & -1         & 1          & 3  \\
4  & a        & 2          & 0          & 3          & 4  \\
5  & b        & -1         & 1          &            &    \\
6  & b        & 1          & 2          &            &    \\
   &          &            &            &            &    \\
7  & b        & 2          & 4          & 3          & 7  \\
8  & a        & 3          & -1         &            &    \\
9  & a        & 3          & 8          &            &    \\
10 & a        & 0          & 3          & 2          & 10 \\
11 & b        & 6          & 2          &            &    \\
12 & a        & 10         & 3          & 2          & 12 \\
13 & b        & 7          & 4          &            &   
\end{tabular}
\end{table}

% Please add the following required packages to your document preamble:
\begin{table}[h]
\begin{tabular}{@{}llllll@{}}
\toprule
i  & S{[}i{]} & NSV{[}i{]} & PSV{[}i{]} & LPF{[}i{]} & LZ \\ \midrule
0  & a        & -1         & -1         & 0          & 0  \\
1  & b        & -1         & 0          & 0          & 1  \\
2  & b        & 1          & 0          & 1          & 2  \\
3  & a        & 0          & -1         & 1          & 3  \\
4  & a        & 2          & 0          & 3          & 4  \\
5  & b        & -1         & 1          &            &    \\
6  & b        & 1          & 2          &            &    \\
   &          &            &            &            &    \\
7  & b        & 2          & 4          & 3          & 7  \\
8  & a        & 3          & -1         &            &    \\
9  & a        & 3          & 8          &            &    \\
10 & a        & 0          & 3          & 2          & 10 \\
11 & b        & 6          & 2          &            &    \\
12 & a        & 10         & 3          & 2          & 12 \\
13 & b        & 7          & 4          &            &    \\ \bottomrule
\end{tabular}
\caption{Split=7,LZ=8}
\end{table}

% Please add the following required packages to your document preamble:
% \usepackage{booktabs}
\begin{table}[h]
\begin{tabular}{@{}llllll@{}}
\toprule
i  & S{[}i{]} & NSV{[}i{]} & PSV{[}i{]} & LPF{[}i{]} & LZ \\ \midrule
0  & a        & -1         & -1         & 0          & 0  \\
1  & b        & -1         & 0          & 0          & 1  \\
2  & b        & 1          & 0          & 1          & 2  \\
3  & a        & 0          & -1         & 1          & 3  \\
   &          &            &            &            &    \\
4  & a        & 2          & 0          & 3          & 4  \\
5  & b        & -1         & 1          & -          & -  \\
6  & b        & 1          & 2          & -          & -  \\
7  & b        & 2          & 4          & 3          & 7  \\
   &          &            &            &            &    \\
8  & a        & 3          & -1         & 2          & 8  \\
9  & a        & 3          & 8          & -          & -  \\
10 & a        & 0          & 3          & 2          & 10 \\
11 & b        & 6          & 2          & -          & -  \\
   &          &            &            &            &    \\
12 & a        & 10         & 3          & 2          & 12 \\
13 & b        & 7          & 4          & -          & -  \\ \bottomrule
\end{tabular}
\caption{Split=4,LZ=9}
\label{my-label}
\end{table}

% Please add the following required packages to your document preamble:
% \usepackage{booktabs}
\begin{table}[h]
\begin{tabular}{@{}lllllll@{}}
\toprule
i  & S{[}i{]} & NSV{[}i{]} & PSV{[}i{]} & PrevOcc{[}i{]} & LPF{[}i{]} & LZ \\ \midrule
0  & a        & -1         & -1         & -1             & 0          & 0  \\
1  & b        & -1         & 0          & -1             & 0          & 1  \\
2  & b        & 1          & 0          & 1              & 1          & 2  \\
   &          &            &            &                &            &    \\
3  & a        & 0          & -1         & 0              & 1          & 3  \\
4  & a        & 2          & 0          & 0              & 2          & 4  \\
5  & b        & -1         & 1          & -              & -          & -  \\
   &          &            &            &                &            &    \\
6  & b        & 1          & 2          & 1              & 3          & 6  \\
7  & b        & 2          & 4          & -              & -          & -  \\
8  & a        & 3          & -1         & -              & -          & -  \\
   &          &            &            &                &            &    \\
9  & a        & 3          & 8          & 3              & 3          & 9  \\
10 & a        & 0          & 3          & -              & -          & -  \\
11 & b        & 6          & 2          & -              & -          & -  \\
   &          &            &            &                &            &    \\
12 & a        & 10         & 3          & 10             & 2          & 12 \\
13 & b        & 7          & 4          & -              & -          & -  \\ \bottomrule
\end{tabular}
\label{my-label}
\caption{Split=3,LZ=8}
\end{table}

The construction of the suffix array stays true to the algorithm used by deo and meely \cite{ }.
The pseudocode for the skew algorithm is presented in.
Smaller individual kernels can be used to first.

To sort the partial suffixes using radix sort, I used the CUB implementation of radix sort.
After those partial suffixes are sorted, they need to be checked for uniqueness.
To calculate the prefix sum, I used the device wide CUB implementation.
Room for improvement.
Transfer back to the cpu to control recursion
Used cub for istriple kernel
ModernGPU merge sort

\section{ANSV}

To calculate the needed ANSV values I used the parallel algorithm from jshun \cite{ }.

The first step is to build a balanced binary tree, where the leaves are elements from SA, and the ancestors are the minima of their children.
Although more efficient algorithms may exist, I decide to take a simpler naive approach and launch a kernel at each level.
Each thread in the kernel calculates for a node the minimum of its two children and stores it into a 1d array.
A 2d array would be easier to index into, but much more difficult to allocate.
(Something about memory locality and indexing into 1d array).

The suffix array is then divided into even divisions.
Each thread uses a stack, in the form of an array, and traditionally solves ANSV for their division.
Because each thread can only see their division, many of the positions will think there is no smaller position, while they may exist in the next or previous division.
To compensate for this, each thread will manually check each position that did not find a smaller value using RMQs on the previously generated binary tree.

This algorithm will generate the ANSV arrays for each index, although not every index is needed in the final LZ factorization. 
I did play around with solving the ANSV problem for a specific index only when needed, but found that in most cases, this was only a little faster or much slower.

\section{LZ Factorization}

The final step is to calculate the LZ factorization.

At first I attempted to follow the parallel algorithm of jshun.
In their work, the LPF array is calculated for every position, and then the LZ factorization is solved using a parallel list ranking algorithm.
Like many more recent works, I found that the calculation of the LPF array at every index to be too computationaly expensive and wasteful, even on the GPU.
Instead, my work will also employ the lazy LZ factorization, mentioned in \cite{ }.
The biggest problem with the lazy LZ factorization was that it is incredibly sequential.
Since it is impossible to know what entries will exist in future points in the LZ factorization, it is a hard problem to parallelize.

I propose breaking away from the ideal LZ factorization and using a BLZ.
I define the ideal LZ factorization to be the original LZ factorization, calculated by starting at the first character and greedily solving the problem from left to right.
This LZ factorization would be what is calculated from the original sequential algorithm, and should be the absolute best, hence ideal.
In BLZ, the string S will be broken into chunks to be worked on individually.
Each thread will be assigned a chunk and traditionally calculate the LZ factorization on it.
The LZ factorization calculated by each thread will be entered unmodified into the final LZ factorization.
The main advantage of this is being able to parallelize the problem, while not incurring too many penalties on the compression ratio.
By doing this, I am also able to limit the amount of work any one thread will do, in an attempt to load balance.
There are several disadvantages that may appear, all of which depend on the original input string.
There is a chance for the BLZ LZ factorization to be larger than the ideal LZ factorization.
There is also an unlikely chance for them to be exactly the same.
I would like to explain the different scenarios to you with an anology.

Let's imagine the LZ factorization of a string to be a stairway with floors.
Generating the LZ factorization is like constructing said stairway and floors.
Each floor represents an entry into the LZ factorization.
Each stair represents a character match in the LPF for the floor beneath it.
A sequential algorithm would have you start at the ground floor.
At that point we know where the LPF is located.
Of course in reality, we know of two possible locations where the LPF might be, but we'll simplify it to one for this anology.
When you are on a floor, you, the builder, walk up a stair for each match.
When the LPF no longer matches, you put a floor to represent a new entry and repeat.
In the end, the number of stairs should match the number of original characters, and the number of floors represent the length of the ideal LZ factorization.

thm: The BLZ LZ factorization >= ideal.
proof:
Now we'll look at what happens when we use BLZ to parallelize the problem.
First, we remove all the floors.
Next, to split up the input, we insert ground floors at regular intervals.
After, we travel the stairway inserting floors, like the sequential algorithm above.
Finally, we stop when we get to the next ground floor.
There are several scenarios that can occur.
The first scenario occurs when an inserted ground floor is inserted where an existing floor used to exist.
The LZ factorization calculated, starting at this ground floor, is exactly the same as the ideal LZ factorization, up until the next ground floor.
The extreme of this scenario happens if every inserted ground floor is inserted where a floor used to exist.
The BLZ factorization is then exactly the same as the ideal LZ factorization.

The next scenarios occurs when the inserted ground floors are inserted in between existing floors from the ideal.
When this happens, the BLZ LZ factorization is automatically increased by at least one, from the new ground floor.
The LPF of the previous floor is no longer the longest it can be.
This also limits the maximum offset for any offset to be the chunk size.
This new ground floor now performs the sequential algorithm as always.
The next floor that is inserted by the builder may or may not be where a floor existed.
This is entirely dependent on the input string and is impossible to predict.
It is very likely though at some point, for the floors to again match the existing floors.
At that point onward, the BLZ LZ factorization again matches with the ideal.

The next question to be answered is deciding the chunk size. 
A larger chunk size could reduce the chances for a larger factorization and increase compression ratios.
On the other hand, a smaller chunk size would more evenly distribute the work among the GPU threads, and in turn should increase compression speeds.
This is a tradeoff that should be left to the user, in my opinion.
In my implementation, I have the option to use arbitrary sizes or even divisions of the input.
Some optimal sizes might be divisions of the input of multiples of the number of multiprocessors.
In any case, it is impossible to predict the compression ratio, and different people will have different priorities.

One thing that I did have yet to cover is how we perform the string match.
The naive operation is to do a character by character match until the prefix no longer matches with the LPF.
One optimization that I have implemented is to instead do a parallel string comparison.
A block of threads can load a group of characters into shared memory.
The BlockLoad primitive from CUB is used to load a number of characters from the LPF and from the prefix into arrays for use in an individual thread in the block.
To simplify, we can imagine a block of 32 threads loading a chunk of 32 characters from the LPF and prefix.
Each thread, responsible for a single index and two characters, then compares the two characters for a match.
A match is assigned the value blockDim.x (32), and a mismatch is assigned the value threadIdx.x (0-31).
The threads can then cooperatively min reduce the problem, with the BlockReduce primitive from CUB for example.
If all 32 characters matched, the value 32 is returned to the first thread in the block, which controls all the logic.
That thread will then set a flag to indicate to the rest of threads to continue with the comparisons.
If there exist a mismatch, the index of the mismatch is instead returned to the first thread, which can then stop the comparisons.
Which implementation is faster is soley dependent on the data and the average factor length.
Average factor lengths less than the number of characters loaded and compared in parallel may see faster speeds with just the naive comparisons.
My implementation will use the parallel string comparison for evaluation.

Finally, it is shown how we can calculated the LPF. 
Because the LPF can occur at either the PSV or the NSV, we need to check both and pick the longer as mentioned before. 
Reaching the edge of the boundary of a chunk in the first search allows us to skip the second search.When the longer is found, we can insert that into a prevocc array. 

The final LZ factorization, made up of relevant entries from the LPF and prevocc arrays, can be gathered by using the DeviceSelect and CudaMemcpy.
