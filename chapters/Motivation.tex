\chapter{Motivation}
The recent influx of commercially available 3D input devices and immersive 3D displays to the consumer market, coupled with high performance consumer graphics hardware, has given end users access to all of the hardware needed to support high quality, immersive 3D human-computer interaction for the first time ever. However, current application support of this kind of hardware is very limited, consisting mainly of demo applications and video games, and applications which support more than one of these devices are even more rare. 

\section{Obstacles Facing the Adoption of 3D Interfaces}

In general, immersive 3D user interfaces require both a good 3D input device and an immersive 3D display, and with such limited device support it is rare that an application supports both and even more rare that an end user will own a 3D input device and an immersive 3D display which are both supported by the 3D user interface application they wish to use (much less every 3D user interface application they wish to use). This problem could be attributed to many factors, including that it is too early in the life of these devices for applications to have been developed, or that there is simply limited application potential for this these devices. And while these certainly could be contributing factors, there are also tangible shortcomings in the software ecosystem surrounding this new hardware which indicate that this problem is not simply going to go away on its own.

\subsection{Device Abstraction}

The first problem to become immediately apparent is the sheer diversity of 3D interface devices, a problem which will only get worse as more and more devices come to market. There is a fair amount of similarity within these devices, and the even greater similarity within the actual 3D interface primitives which each is actually capable of providing. Every 3D input device discussed above provides either some information about the 3D pose of the users body [cite Jester] or the 3D transform of some kind of handheld device, and immersive 3D displays all serve the same purpose of giving the user a sense of presence in a virtual (or mixed reality) 3D space. Despite this, there is no widely adopted abstraction for either 3D input devices or immersive 3D displays, and while some such abstractions exist, each has its own unique shortcomings (this is discussed further in the related works section).

Rather, every one of these devices comes with its own API, designed to work with that device and usually other devices from the same manufacturer. If an application wishes to use this device it must be ported to use that device’s API, and if it wishes to be compatible with multiple devices in the same class from different manufacturers it must include dedicated code for each of the devices it wishes to support, and this code must be maintained by the developers of each application independently. Support for devices can be abstracted by a user interface toolkit like Vrui [citation], a video game engine like Unity or Unreal (via vendor provide plugins), or even a dedicated abstraction libraries like MiddleVR [citation] or VRPN [citation]. Each of these has its own strength and weaknesses, but there are also overarching shortcomings of including the abstraction layer in toolkits used on a per application basis. First, this means that the device abstraction layer has to be replicated for each toolkit (causing the same problems as replicating it for each application). This could hypothetically be resolved by the uniform adoption of a single toolkit which meets the need of every application needing a 3D user interface, but given the wide variance in demands between something like a 3D file browser and an immersive VR experience, this seems both unrealistic and, in the author’s opinion, very much undesirable. Secondly, if the abstraction is done within toolkits used on a per application basis, then two applications using different toolkits (or perhaps even the same toolkit) that attempt to use the same device simultaneously could block one another from accessing the device. This is closely related to the next major problem with the software ecosystem surround 3D user interface devices.

\subsection{Multiple Application Support}

The ability to use multiple applications simultaneously has become a core feature of the interface paradigms we use today, and the ability of a user to install and run together whichever set of applications they like is the key point of software modularity that has allowed personal computers to be useful to a broad class of users with highly diverse requirements.
 
This same point of modularity can be applied to 3D user interfaces, and to a certain extent it already is. It is certainly possible to install multiple 3D user interface applications and, depending on the applications, maybe even run them simultaneously. However, there are serious limitations here as well, particularly when it comes to immersive 3D displays. These displays require custom projection of a 3D scene for each eye, and this projection must be consistent with the 3D position of the users head relative to this scene and the display surface, and many HMDs require a post-projection adjustment to correct for distortion introduced by the optical system (this is discussed in detail in the background section). While it is relatively straightforward to implement this behavior in an application which draws itself (and only itself) to the entire display surface, sharing the 3D display between multiple applications with 3D user interfaces introduces significant problems.
The essential problem is that in order for the 3D interface space to be divided between multiple 3D interfaces from different applications in a meaningful way, it must be divided in 3D. This is difficult because current graphics and windowing infrastructure, as well as the 2D display technology underlying the immersive 3D display, is designed around the paradigm of applications producing 2D output which is combined in 2D by the windowing system and driven onto the 2D interface space of a traditional display. This works well for dividing the 2D space of a display among multiple 2D interfaces, since they can each be given a rectangular region of the rectangular display, but dividing the 2D display surface of an immersive 3D display among multiple 3D interfaces in the same way (without applying the correct stereo projection and optical distortion correction) produces results which do not appear to be in the same 3D space. 

This means that while an immersive 3D display can produce a compelling 3D interface space for a single application, it is not possible for multiple 3D interface applications to share the 3D display in the same way that 2D applications can share a 2D display. It also means that 2D applications which have no reason to need a 3D interface are also unable to use the immersive display, despite the fact that embedding a 2D interface surface in a 3D interface space is conceptually simple and well defined.

\section{Insights from Two Dimensional User Interfaces}

	The core goal of this thesis is derived from the simple observation that the problems currently facing the development of applications with 3D user interfaces and the integration of the hardware that supports them are present for 2D interfaces as well, with the key difference that in the domain of 2D interfaces these problems have already been solved. Despite the fact that the diversity of displays, mice, and keyboards dwarfs the diversity of 3D user interface devices, users are able to assemble a hardware interface from almost any combination of devices they like and run all of their favorite applications on top of their custom hardware interface. New 2D interface devices need not be integrated into every application that uses them, and multiple 2D interfaces from different applications can be used together in the same 2D interface space in arbitrary combinations. 
	
\subsection{Windowing Systems}

	Applications with 2D interfaces no longer suffer these problems is because modern consumer operating systems provide a set of 2D interface abstractions called a windowing system. Windowing applications do not interact directly with the mouse, keyboard, or display. Rather the windowing system manages input devices and displays (usually through lower level abstractions provided by the kernel), and provides the interface capabilities of these devices as services to applications. Applications receive input events like mouse movement from the windowing system abstractly without needing any knowledge of what type of mouse is used, how it is connected, or who manufactured it. The 2D images produced by applications are not drawn directly to the display, they are given to the windowing system which then composites the output of all running applications (sometimes in a separate compositor program, depending on the windowing system) into a final 2D image which is scanned out to the display itself. 
This basic system architecture is present, with slight variation, in every major graphical operating system. It is connected with the prevalent ‘Windows, Icons, Menus, Pointer’ (WIMP) interaction paradigm and the popular desktop metaphor, which are well understood, well tested, and familiar to users. This architecture has also strongly influenced the way applications interact with hardware accelerated 3D graphics systems, leading to a design pattern where applications are responsible for projecting their 3D content into a 2D image before delivering it to the windowing systems, and this has in turn influenced both the design of 3D graphics API’s as well as the hardware which underlies them. The ubiquity of windowing systems has also profoundly affected the high level topology of the software ecosystem surrounding WIMP interaction, leading to the emergence of user interface toolkits like QT and Java Swing that abstract popular windowing systems behind their common functionality so that sophisticated, cross platform, WIMP applications can be developed without knowledge of the underlying software mechanisms, much less the hardware devices, that support them.
 
Even existing 3D interface applications use the windowing system to abstract traditional input and to draw to the 2D display that underlies its immersive 3D display, but without the ability to abstract 3D input devices and to mix 3D interfaces in 3D, these windowing systems do not give applications the ability to share 3D interface hardware in a meaningful way.

\section{Proposed Solution: A Three Dimensional Windowing System}

The primary goal of this thesis is to demonstrate that windowing systems are capable of solving the some of the core problems facing 3D user interfaces in the same way that they have already solved the exact same problems for 2D user interfaces, and that this can be done with extensions to an existing windowing system, allowing both unmodified 2D applications and as device-agnostic 3D applications to window into the same 3D interface space. 

The type of windowing system described here extends the concept of a window as a 2D region of a 2D interface space to the 3D interface space provided by the 3D user interface hardware described above. It allows 3D applications to create a 3D analog of a traditional window, representing a 3D region of the 3D interface space which can be manipulated in 3D in much the same way as a traditional 2D window can be manipulated in 2D. These 3D windows can listen for 3D input events via the same mechanism that is used to listen to 2D input events, and the 3D output they produce is mixed in 3D with the 3D output of other 3D applications.

Additionally, this type of windowing system allows traditional, unmodified 2D applications to create a 2D interface context in this 3D windowing space which behave exactly the same as a normal window from the applications perspective. The windowing system embeds these 2D windows in the space in much the same way that a sheet of paper embeds a 2D document in 3D reality, allowing the user to manipulate and manage these windows as 3D objects. 3D input events managed by the windowing system are projected onto the 2D window before being delivered to the 2D application, allowing the user to send meaningful 2D input to the application with a 3D input device.

\subsection{Advantages of This Approach}

There are numerous advantages to solving these problems for 3D user interfaces in the same way that we solve them for 2D interfaces, a few of which are discussed here in detail. Some of these advantages are the same advantages that led to the adoption of windowing systems in the first place, and others are simply result from leveraging extensive work put into windowing systems for 2D interfaces. This is by no means meant to be an exhaustive list.

\subsubsection{Hardware Abstraction and Multiple Application Support}

This approach allows a hardware 3D interface (consisting of at least one immersive 3D display and at least one 3D input device) to support a unified 3D interface space, where both 2D and 3D applications are treated as first class components of the human-computer interface and managed together in the 3D space via a unified window management mechanism. 

This means that any hardware capable of supporting the 3D windowing system can support all 3D applications which use it (as is the case with 2D applications), and that new hardware need only provide a driver for the windowing system abstraction to achieve support from all applications using the system (as is also the case with 2D interfaces). 

It also means that the structure of the software ecosystem surrounding 2D WIMP interaction can be applied to the software ecosystem surrounding 3D interfaces, allowing the development of a wide variety of user interface toolkits which provide high-level, domain-specific interaction metaphors built on top of common abstractions provided by the windowing system, allowing multiple applications using different toolkits (or no toolkit at all) to share the 3D interface hardware supporting the system in a meaningful way. Furthermore, because the system supports unmodified 2D applications, support for 3D interface elements could even be integrated into existing 2D interface toolkits where appropriate.

\subsubsection{Compatibility With Existing Graphics and Windowing Infrastructure}

	As the provided implementation demonstrates, it is possible to support compositing 3D content in a 3D space while only needing to send 2D images from the application to the windowing system. This means that existing, full-featured 3D graphics API’s, which give the application full control over every aspect of how its 3D content is drawn into a 2D image, are still perfectly well suited to this task. This means that applications retain full flexibility in what they draw and how they draw it, and can still benefit from acceleration by high performance consumer graphics processing units (GPUs). It also means that 3D applications still benefit from the extensive software infrastructure that has been put in place to allow 2D applications to efficiently pass 2D images to the windowing system and to allow the windowing system to composite these images off screen. Together this means that the 3D windowing system can efficiently support both 2D and 3D applications without needing to lay extensive new infrastructure to do so.
