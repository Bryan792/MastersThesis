\chapter{Design: A Unified Windowing System}
The core goal of this thesis is to demonstrate, both conceptually and practically, that windowing systems are capable of solving the same problems for 3D user interfaces that they currently solve for 2D user interfaces, that a single windowing system can solve the same problems for 2D and 3D interfaces, and that a system which does so can be built on top of an existing windowing system. To understand this further, let’s examine abstractly the services that a windowing system provides and how these map onto 3D interfaces.

\section{Windowing System Services}

In general, windowing systems provide a software platform for graphical applications that gives these applications a means to use the hardware resources they need to provide graphical interfaces without needing to interact with the hardware directly. Because the windowing system handles the direct interactions with the hardware, it is able to multiplex the use of hardware resources between many individual applications which need the capabilities they provide. Because providing these hardware capabilities to applications abstractly is the core purpose of the windowing system, it is important to understand what it is that they represent.

\subsection{Graphical Interface Hardware and The Graphical Interface Space}

Consider the interface hardware needed to provide a traditional, two dimensional, WIMP interface (and thereby needed to support a traditional windowing system). There are three essential components: a display, a mouse, and a keyboard. The display provides a two dimensional space in which two dimensional images can be drawn, and  the mouse allows limited symbolic input at any point in this two dimensional space. Together these two devices create a two dimensional spatial interface, two dimensional input and output in the same two dimensional space. 

The extension of this concept to three dimensions requires the ability of the hardware system to support three dimensional input and three dimensional output in the same three dimensional space, creating a proper three dimensional spatial interface. Immersive 3D displays provide the user with a compelling illusion of a three dimensional space which the computer can fill with arbitrary 3D content, and 3D input devices can be used to provide 3D input in this space, so existing, consumer grade 3D interface hardware can provide such a 3D interface space in the same way that a mouse and traditional display provide a 2D interface space. If the choice of 3D input device is restricted to hand held tracking devices with buttons, like the Razer Hydra used in the implementation presented here, then the 3D input set it provide is very closely analogous to the 2D input set provided by a traditional mouse: symbolic input from a few buttons coupled with continuous spatial tracking throughout the interface space.

The keyboard allows the user to give the computer complex symbolic input, but there is no spatial information attached to this input, and it is up to the software system to determine which region of the interface space, if any, that this input is delivered to. The keyboard itself is not related to the number of dimensions (or any other aspect) of the interface space, and this means that it can continue to serve its purpose in a 3D interface without needing to undergo any changes to its hardware design.

\subsection{Interface Contexts Within the Graphical Interface Space}
	


\section{Three Dimensional Windows With Two Dimensional Buffers}
\subsection{View And Projection Matrices}
\subsection{Stereo Color Buffers}
\subsection{Depth Buffers}
\subsection{Clipping}