\chapter{Contribution}
The primary contribution of this work is an open source implementation of a 3D windowing system built on top of the Wayland display server protocol. It is intended both to demonstrate that windowing systems can solve some of problems hindering the adoption of 3D user interfaces, as well as to provide a body of code capable of forming the core of a functioning, full featured, open source 3D windowing system. 
	
This implementation includes the Wayland protocol extensions necessary to enable 3D windowing, a framework for building Wayland compositors which support these extensions (built on top of the QtWayland Compositor library), an example compositor which uses this framework to support the windowing system on top of the Oculus Rift Developer Kit HMD and the Razer Hydra motion controller, drivers for these devices, a client side library for handling the protocol extensions and graphics trickery needed for 3D windowing, and a few example clients which  demonstrate how to use the client side library. 

This software demonstrates the ability of consumer 3D interface hardware to support a 3D windowing system, and the ability of this 3D windowing system to support applications with compelling 3D interfaces. It also demonstrates that this style of windowing system can be built on top of existing graphics and windowing infrastructure, and that if can support unmodified 2D applications windowing into the same 3D interface as the 3D applications. 

This implementation is not intended to be release quality by the completion of this thesis, and it is not intended to embody all of the functionality which such a windowing system could hypothetically provide, particularly when it comes to device abstraction. Rather, it is intended to show what is possible, and provide the core functionality needed in a piece of software which is modular enough to form the core of a comprehensive, open source solution. 
