
% Cal Poly Thesis
% 
% based on UC Thesis format
%
% modified by Mark Barry 2/07.
%

\documentclass[12pt]{ucthesis}

\usepackage{ifpdf} 
\newif\ifpdf
\ifx\pdfoutput\undefined
    \pdffalse % we are not running PDFLaTeX
\else
\pdfoutput=1 % we are running PDFLaTeX
\pdftrue \fi

\usepackage{url}
\usepackage{multicol}
\ifpdf

    \usepackage[pdftex]{graphicx}
    % Update title and author below...
    \usepackage[pdftex,plainpages=false,breaklinks=true,colorlinks=true,urlcolor=blue,citecolor=blue,%
                                       linkcolor=blue,bookmarks=true,bookmarksopen=true,%
                                       bookmarksopenlevel=3,pdfstartview=FitV,
                                       pdfauthor={Forrest Reiling},
                                       pdftitle={Extending Windowing Systems to Three Dimensions},
                                       pdfkeywords={thesis, masters, cal poly}
                                       ]{hyperref}
    %Options with pdfstartview are FitV, FitB and FitH
    \pdfcompresslevel=1

\else
    \usepackage{graphicx}
\fi

\usepackage{hyperref}
\hypersetup{
	linktoc=all,
    colorlinks,
    citecolor=black,
    filecolor=black,
    linkcolor=black,
    urlcolor=black
}



\usepackage{titlesec}
% \titleformat{\chapter}[display]% OLD
%     {\normalfont\huge\bfseries}{\chaptertitlename\ \thechapter}{20pt}{\Huge}% OLD
% \titlespacing*{\chapter}{0pt}{50pt}{40pt}% OLD
\titleformat{\chapter}[display]% NEW
    {\normalfont\centering}{\chaptertitlename\ \thechapter}{12pt}{}% NEW
\titlespacing*{\chapter}{0pt}{30pt}{20pt}% NEW

%\titleformat{\section}[block]{first}{label}{12pt}

\titleformat{\section}{}{\thesection}{1em}{}
\titleformat{\subsection}{}{\thesubsection}{1em}{}
\titleformat{\subsubsection}{}{\thesubsubsection}{1em}{}
\titleformat{\paragraph}{}{\theparagraph}{1em}{}

\usepackage[font={}]{caption}

%\renewcommand{\cftchapleader}{\cftdotfill{\cftdotsep}} % for chapters
%\renewcommand{\cftsecleader}{\cftdotfill{\cftdotsep}} 


\usepackage{amssymb}
\usepackage{amsmath}
\usepackage[letterpaper]{geometry}
\usepackage[overload]{textcase}
%\usepackage[toc,page]{appendix}

\usepackage{tabularx}
\usepackage{mdframed}
\usepackage{algpseudocode}
\usepackage{booktabs}
\usepackage{fixltx2e}


\usepackage{rotating}

\usepackage{enumitem}
\setlist{nolistsep}


\usepackage{float}
\floatstyle{boxed}
%\restylefloat{table}

\bibliographystyle{abbrv}

\setlength{\parindent}{0.25in} \setlength{\parskip}{6pt}

\geometry{verbose,nohead,tmargin=1.25in,bmargin=1in,lmargin=1.5in,rmargin=1.3in}

\setcounter{tocdepth}{4}
\setcounter{secnumdepth}{4}

% Different font in captions (single-spaced, bold) ------------
%\newcommand{\captionfonts}{\small\bf\ssp}
\newcommand{\captionfonts}{}

\makeatletter  % Allow the use of @ in command names
\long\def\@makecaption#1#2{%
  \vskip\abovecaptionskip
  \sbox\@tempboxa{{\captionfonts #1: #2}}%
  \ifdim \wd\@tempboxa >\hsize
    {\captionfonts #1: #2\par}
  \else
    \hbox to\hsize{\hfil\box\@tempboxa\hfil}%
  \fi
  \vskip\belowcaptionskip}
\makeatother   % Cancel the effect of \makeatletter
% ---------------------------------------



\begin{document}

% Declarations for Front Matter

% Update fields below!
\title{Optimizing Lempel-Ziv Factorization for the GPU Architecture}
\author{Bryan Ching}
\degreemonth{June} \degreeyear{2014} \degree{Master of Science}
\defensemonth{June} \defenseyear{2014}
\numberofmembers{3} \chair{Professor Chris Lupo, Ph.D.,\newline Department of Computer Science} \othermemberA{Professor John Seng, Ph.D.,\newline Department of Computer Science} \othermemberB{Professor Zachary N J Peterson, Ph.D.,\newline Department of Computer Science} \field{Computer Science} \campus{San Luis Obispo}
\copyrightyears{seven}



\maketitle

\begin{frontmatter}

% Custom made for Cal Poly (by Mark Barry, modified by Andrew Tsui).
\copyrightpage

% Custom made for Cal Poly (by Andrew Tsui).
\committeemembershippage

\begin{abstract}
Lossless data compression is used to reduce storage requirements, allowing for the relief of I/O channels and better utilization of bandwidth.
The Lempel-Ziv lossless compression algorithms form the basis for many of the most commonly used compression schemes.
General purpose computing on graphic processing units (GPGPUs) allows us to take advantage of the massively parallel nature of GPUs for computations other that their original purpose of rendering graphics. 
Our work targets the use of GPUs for general lossless data compression.
Specifically, we developed and ported an algorithm that constructs the Lempel-Ziv factorization directly on the GPU.
Our implementation bypasses the sequential nature of the LZ factorization and attempts to compute the factorization in parallel.
By breaking down the LZ factorization into what we call the PLZ, we are able to outperform the fastest serial CPU implementations by up to 24x and perform comparatively to a parallel multicore CPU implementation.
To achieve these speeds, our implementation outputted LZ factorizations that were on average only 0.01 percent greater than the optimal solution that what could be computed sequentially.

We are also able to reevaluate the fastest GPU suffix array construction algorithm, which is needed to compute the LZ factorization.
We are able to find speedups of up to 5x over the fastest CPU implementations.

\end{abstract}


\begin{acknowledgements}

Thanks to:

\begin{itemize}
\item My advisor Chris Lupo, for all of his guidance and patience and for introducing me to the GPGPU.
\item My parents, my family, and everyone who has supported me to where I am today.
\end{itemize}

\end{acknowledgements}

\tableofcontents

\listoftables

\listoffigures

\end{frontmatter}

\pagestyle{plain}

\renewcommand{\baselinestretch}{1.66}


% ------------- Main chapters here --------------------
\chapter{Introduction}
Lossless data compression has the ability to reduce storage requirements, while still maintaining the integrity of the original data.
Several advantages can be gained by reducing the size of data, including the relief of transfer accoss I/O channels.
Compression algorithms have a tradeoff, in that they require an additional computation to be done on the original data before a compressed version can be used.
This can be computationally expensive and the cost to compress might require too much processing or time.
In many cases and applications, the increase of bandwidth rates outweighs any other consideration, but the increase in compression rates would generally be helpful.
This work takes a look into speeding up those compression rates by performing the compression directly on a GPU, a graphics processing unit.

An increase of applications and algorithms are being developed to utilize the relatively new general purpose computing (GPGPU) apsect of GPU technology.
GPGPUs allow applications to run computations unrelated to graphics, while allowing for the exploitation of the massively parallel nature of GPUs.
GPGPUs are becoming increasingly popular for high performance computing, and many of the world's fastest supercomputers utilize GPGPUs in large clusters.

GPGPU as coprocessor
GPGPU massively parallel
difference vs CPU

Thesis statement
Organization

\chapter{Background}

\section{GPU Architecture}

\subsection{CUDA}
\subsection{Memory Model}
\subsection{Thread Model}
\subsection{Libraries and Parallel Primitives}

There exist a variety of libaries that make development on CUDA more streamlined.
These libraries provide a variety of uses for the CUDA programmer.
Some provide a fast solution to a particular problem.
Others provide layers of abstraction to hide the complexity of CUDA programming.
This includes the transfer of memory and the thread/block/grid handling.
In our implementation, we try and use libraries whenever possible.
First, these libraries have been developed over a long time by people who are more familiar with the architecture and the quirks that come with it.
Second, abstraction allows a problem to be continually optimized, while presenting a common API to use.
This allows us to somewhat futureproof our implementation, since the libraries should be updated in the future against newer CUDA versions and hardware.
Also, some problems may seem simple at first, but too complex for a user to implement every time.
Lastly, the purpose of many libraries are to provide solutions to parallel primitives.

Parallel primitives change the way a programmer looks at implementing a parallel algorithm on the GPU.
Instead of having to create a totally new algorithm specific for the GPU, one can change their program to be a collection of parallel primitives.
These parallel primitives are common operations that we see in parallel algorithms across any architecture.
Some very well known operations are scans, like prefix sum, or reductions, such as finding the min or sum of an array.
Although these primitives may seem simple at first, there are many tricks used by these libraries to provide speed up.
Many of these are CUDA specific and a beginner to intermediate CUDA engineer are likely to not know them.

One of the most widely used CUDA libraries is the Thrust library.
The Thrust library claims to resemble the STL and provide device-wide primitives to be used.
One of the key features of the Thrust library is the interoperability of different architectures and technologies (CUDA, OpenMP, TBB).
Although this may be nice for portability, we decided to avoid the use of Thrust and use the more CUDA-specific CUB library.
CUB provides abstractions at all three layers of the CUDA thread model, the device, the block, and the warp.
CUB is more aware of CUDA features, like streams.
That said, many of the algorithms are shared between CUB and Thrust.
We decided to choose CUB for its claims at higher performance than Thrust and specific features unavailable in Thrust, like the primitives that work on the block level.
Some of the primitives that we use in our implementation include an inclusive sum, device select, radix sort, and block reduce.

Another interesting library that we make use of is Modern GPU, MGPU.
More specifically we make use of its algorithm for merge sorting.
One of the primary concerns when parallel programming is how to load balance a problem across the threads.
MGPU makes use of a technique called Merge Path to achieve this \cite{ }.
Merge Path realizes that if we imagined the two arrays on a grid, the merge sort follows a path through it.
We can run diagonals throught the grid to find their intersection with this Merge Path, where all comparisons are true on one side and false on the other.
This can be done quickly using a binary sort.
We can now figure out exactly which sections of the arrays correspond to sections of the final merged array.
With even divisions, we can evenly divide the final output among the threads, each knowing which sections of the input array they need.
Those threads can then merge those sections, employing Merge Path if they so wish to.

There are some caveats when using these libraries.
The first is that they provide an additional dependency to your application.
Sometimes these libraries can be cumbersome to install or too big for the host system.
The next is that it is feasible to write a higher performant code with specific knowledge of the application.
For example, knowing that part of the input array always appears in certain positions in the merged output could be utilized by the programmer.
We decide to ignore these caveats in our implementation.

\section{Compression}

talk about compression and lempel ziv
lossy vs lossless

\subsection{Lempel Ziv Factorization}

The LZ factorization of a string S[n] decomposes S into factors S = w1w2wk where k<=n, where each factor wi is either the longest factor that appears left of wi in S or is a new character.
For example, the LZ factorization of string abbaabbbaaabab has the factorization a.b.b.a.abb.baa.ab.ab.
Various algorithms have been compared experimentally in \cite{ }.
In general, LZ factorization algorithms all make use of a few common data structures and stages, the suffix array, the LCP array, the LPF array, and the PrevOcc array.

\subsubsection{Suffix Array}

The suffix array is a common data structure in string matching algorithms.
The suffix array SA of S is a lexicicographically ordered array of integers of size n where each integer represents a suffix of S, so that suf[SA[0]] < suf[SA[1]] < . . . suf[SA[n-1]].

First introduced as space efficient alternative to suffix trees, the suffix array can fully replace the suffix tree with the use of additional data structures, such as the LCP array.
Suffix arrays can be used to quickly find and match strings in a dictionary.
This ability has a wide variety of applications from string searches to data compression to bioinformatics.

There exist many suffix array construction algorithms (SACA).
The skew algorithm of Kark \cite{ } uses a divide and conquer approach to construct a partial suffix array to infer the rest of the positions.
The pseudocode of the SACA that we will use is presented in Figure.
Essentially, the skew algorithm divides the suffixes into two groups.
A suffix array is constructed using the larger group, which holds 2/3 of the suffixes.
A quick check is used to see if these suffixes can be quickly sorted using their first three characters.
If this sort does not create the suffix array, due to non-unique suffixes, then the algorithm recurses until the suffix array is constructed.
The smaller group can then be sorted using inference and merged with the larger group.
Running in linear time, the skew algorithm has also been studied in parallel.
The fastest known construction of suffix arrays on the GPU by Meo and Deeeley utilizes the skew algorithm.
Inspired by most of their ideas, our work is also a reimplementation and benchmark of their algorithm.

\subsubsection{LCP and LPF Array}

The LCP, longest common prefix, array is an auxiliary structure to the suffix array that provides the longest common prefix between successive suffixes in SA.
Formally, position i in the LCP array, LCP[i] = lcp(suf[SA[i-1]],suf[SA[i]]).
The LCP array can be expensive to compute, but recent algorithms have found that the LCP array can be constructed during the construction of the suffix array.

The LPF, longest previous factor, array holds the lengths of the longest factors at any position i.
In other words, LPF[i] holds the maximum lcp of suf[sa[i]] and all suffixes less than i.

\subsubsection{LZ Factorization Calculation}

A naive LZ factorization algorithm may work by calculating the LPF for every position, by calculating the lcp with every previous suffix.
It can be seen that this naive algorithm runs in O(n3) time.
This can be bounded to O(n2) time using the knowledge that the total length of all lcp's is N.
In \cite{ }, the number of positions that a suffix needs to be lcp against is reduced to the PSV and NSV.
%todo talk about why
We only need to check longer suffixes, number smaller, and only the closest ones since lexi order.
The NSV, next smaller value, and PSV, previous smaller value, make up the ANSV, all nearest smaller values, problem.
Computing the ANSV problem can be done linearly and sequentially using a stack-based algorithm found in CITE GABOW.
This observation now reduces the problem to O(n) time.

With these NSV and PSV arrays, a naive algorithm may try and calculate the LPF for every position.
With the LPF array filled at every position, the LZ factorization can be quickly found \cite{ }.
More recent LZ factorization algorithms have decided to forgo this intermediate step and directly calculate the LZ factorization.
If we examine Table X, we can see that the LZ factorization of the string abbaabbbaaabab can be reduced to 8 positions.
The positions where a factor does not begin, position 5 for example, does not need to calculate the LPF.
It also does not need to calculate the ANSV, but that calculation is relatively inexpensive and may come from the generation of the other values anyway.
Referred to as lazy LZ factorization in \cite{Linear Time Lempel-Ziv Factorization: Simple, Fast, Small}, the LPF value is only calculated at the start of a factor.
The recent algorithms \cite{} make use of this fact for simplicity and speedup.
Figure shows the basic pseudocode for calculating the LZ factorization given the PSV and NSV arrays using this lazy method.
The PrevOcc array simply holds the position or suffix where the LPF occurs.

Finally, the LZ factorization can be encoded simply with a position of previous occurence and the length of the match or a character, if the length is 0.
It can also be encoded using the pair of position and previous occurance.
Practical compression schemes might be encoded in triplets with the position, length, and the first letter of mismatch.

\begin{figure}
\begin{algorithmic}[1]
\Procedure{LazyLZFactorization}{$S,n,PSV,NSV$}
\State $i \gets 1$
\While{$i \leq n$}
\State a LZ factor starts here
\If{$lcp(i,PSV) \geq lcp(i,NSV)$}
\State LZ Factor = (PSV,lcp(i,PSV))\Comment{Pair (PrevOcc,LPF)}
\Else
\State LZ Factor = (NSV,lcp(i,NSV))
\EndIf
\State if LPF = 0, PrevOcc = -1, and the character is inserted
\State $i \gets i + max(LPF,1)$\EndWhile
\EndProcedure
\end{algorithmic}
\caption{LZ Factorization}\label{euclid}
\end{figure}

\section{Related Works}

Lossless data compression on the GPU is a field that has yet to be fully investigated.
Many lossless data compression algorithms are application specific.
%“Parallel variablelength encoding on gpgpus,”

Although few, there does exist work on porting general purpose lossless data compression algorithms to CUDA.
CULZSS ports the LZSS algorithm, a sibling to LZ77, to the GPU with success.
The key observation on these ports is the use of pipelining.
Most if not all of these ports split up the data to run their individual algorithm on.
Many of their original algorithms allow for this.
Benefits can be found using CUDA streams to concurrently copy partial data and running kernels.
This is a feature not available in LZ factorization.
Many of these applications also make use of a sliding window, as seen in most LZ77 implementations.
Don't know whole input.
This could lead to larger compressed files.

To the best of our knowledge, this is the first attempt to calculate the LZ factorization on the GPU.
Although we will not be calculating the ideal LZ factorization, as described later, the knowledge of the whole input string is still utilized.
The project by Jshun \cite{ } is a multicore CPU parallel implementation of the LZ factorization.
They provide one of the first and most recent parallel implementations of the LZ factorization.
They provide many of the inspirations throughout our project and implementation.
In their project, they were able to show a O(n) work algorithm with significant speedups on a multicore CPU.
Most of our implementation matches their's, except on the GPU; however, we do not calculate the whole LPF string, and make use of the lazy LZ factorization technique described earlier.
The cost to calculate the ideal LZ factorization in a parallel fashion, as they did, was too great for GPU.
Calculating every LPF position was a very memory intensive task that we found took too long on the GPU due to the high memory latency.
This was the primary cause to the usage of the BLZ, which we'll describe later.

% http://delivery.acm.org.ezproxy.lib.calpoly.edu/10.1145/2380000/2379781/a5-al-hafeedh.pdf?ip=129.65.23.208&id=2379781&acc=ACTIVE%20SERVICE&key=F26C2ADAC1542D74%2E2870C5A035FC0FDB%2E4D4702B0C3E38B35%2E4D4702B0C3E38B35&CFID=343020533&CFTOKEN=37720775&__acm__=1400733002_66a60490d28a219a079b67ccd7c8315e

\chapter{Implementation}
\label{chap:implementation}

Our implementation is structured in three main steps: the construction of the suffix array, the calculation of ANSV for every index, and finally the generation of the LZ factorization.

\section{SA}

\begin{figure}[h]
\begin{algorithmic}[1]
\Procedure{computeSA}{$s,sa,n$}
\State initMod12();\Comment{Kernel to set flags at 2/3. DeviceSelect to get s12,sa12}
\State radixSort($s12$);\Comment{DeviceRadixSort}
\State radixSort($s12$); 
\State radixSort($s12$); 
\State lexicRankOfTriplets();\Comment{Custom kernel to check unique. Inclusive Sum to count. Custom kernel to get s12.}
\If{!allUniqueRanks}
\State computeSA($s12,sa12$);\Comment{Recursion}
\State storeUniqueRanks();\Comment{Kernel}
\Else
\State computeSAFromUniqueRank();\Comment{Kernel}
\EndIf
\State radixSort($s0$); 
\State mergeSort($s0,s12$);\Comment{Merge Path + Merge Sort} 
\EndProcedure
\end{algorithmic}
\caption{Suffix Array Construction Pseudocode from \cite{Deo}. Comments add details from our implementation.}
\label{algorithm:sa}
\end{figure}

\begin{table}[h]
\centering
\begin{tabular}{@{}llllll@{}}
\toprule
We  & S{[}i{]} & NSV{[}i{]} & PSV{[}i{]} & LPF{[}i{]} & LZ \\ \midrule
0  & a        & -1         & -1         & 0          & 0  \\
1  & b        & -1         & 0          & 0          & 1  \\
2  & b        & 1          & 0          & 1          & 2  \\
3  & a        & 0          & -1         & 1          & 3  \\
4  & a        & 2          & 0          & 3          & 4  \\
5  & b        & -1         & 1          &            &    \\
6  & b        & 1          & 2          &            &    \\
   &          &            &            &            &    \\
7  & b        & 2          & 4          & 3          & 7  \\
8  & a        & 3          & -1         &            &    \\
9  & a        & 3          & 8          &            &    \\
10 & a        & 0          & 3          & 2          & 10 \\
11 & b        & 6          & 2          &            &    \\
12 & a        & 10         & 3          & 2          & 12 \\
13 & b        & 7          & 4          &            &    \\ \bottomrule
\end{tabular}
\caption{Split=7,LZ=8}
\label{tab:example7}
\end{table}

\begin{table}[h]
\centering
\begin{tabular}{@{}llllll@{}}
\toprule
We  & S{[}i{]} & NSV{[}i{]} & PSV{[}i{]} & LPF{[}i{]} & LZ \\ \midrule
0  & a        & -1         & -1         & 0          & 0  \\
1  & b        & -1         & 0          & 0          & 1  \\
2  & b        & 1          & 0          & 1          & 2  \\
3  & a        & 0          & -1         & 1          & 3  \\
   &          &            &            &            &    \\
4  & a        & 2          & 0          & 3          & 4  \\
5  & b        & -1         & 1          & -          & -  \\
6  & b        & 1          & 2          & -          & -  \\
7  & b        & 2          & 4          & 3          & 7  \\
   &          &            &            &            &    \\
8  & a        & 3          & -1         & 2          & 8  \\
9  & a        & 3          & 8          & -          & -  \\
10 & a        & 0          & 3          & 2          & 10 \\
11 & b        & 6          & 2          & -          & -  \\
   &          &            &            &            &    \\
12 & a        & 10         & 3          & 2          & 12 \\
13 & b        & 7          & 4          & -          & -  \\ \bottomrule
\end{tabular}
\caption{Split=4,LZ=9}
\label{tab:example4}
\end{table}

\begin{table}[h]
\centering
\begin{tabular}{@{}lllllll@{}}
\toprule
We  & S{[}i{]} & NSV{[}i{]} & PSV{[}i{]} & PrevOcc{[}i{]} & LPF{[}i{]} & LZ \\ \midrule
0  & a        & -1         & -1         & -1             & 0          & 0  \\
1  & b        & -1         & 0          & -1             & 0          & 1  \\
2  & b        & 1          & 0          & 1              & 1          & 2  \\
   &          &            &            &                &            &    \\
3  & a        & 0          & -1         & 0              & 1          & 3  \\
4  & a        & 2          & 0          & 0              & 2          & 4  \\
5  & b        & -1         & 1          & -              & -          & -  \\
   &          &            &            &                &            &    \\
6  & b        & 1          & 2          & 1              & 3          & 6  \\
7  & b        & 2          & 4          & -              & -          & -  \\
8  & a        & 3          & -1         & -              & -          & -  \\
   &          &            &            &                &            &    \\
9  & a        & 3          & 8          & 3              & 3          & 9  \\
10 & a        & 0          & 3          & -              & -          & -  \\
11 & b        & 6          & 2          & -              & -          & -  \\
   &          &            &            &                &            &    \\
12 & a        & 10         & 3          & 10             & 2          & 12 \\
13 & b        & 7          & 4          & -              & -          & -  \\ \bottomrule
\end{tabular}
\caption{Split=3,LZ=8}
\label{tab:example3}
\end{table}

The construction of the suffix array stays true to the algorithm used by Deo and Meely \cite{Deo}.
In Figure~\ref{algorithm:sa}, we show the pseudocode used by Deo and Meely.
To sort the triples, we used CUB's implementation of radix sort.
In line 6, we need to check if the sorted triplets are unique.
To accomplish this, we used a combination of small custom kernels and CUB primitives.
Finally as mentioned in Section~\ref{sec:primitives}, we used the MGPU library to facilitate the merge sort.

\section{ANSV}

\begin{figure}
\begin{algorithmic}[1]
\Procedure{ANSV}{}
\For{each level of MinTree}\Comment{Bottom Up Construction}
\State MinTree($level$);\Comment{Build level by calculating minima of children}
\EndFor
\State ANSVKernel($mintree,chunkSize$)
\EndProcedure
\end{algorithmic}
\begin{algorithmic}[1]
\Procedure{ANSVKernel}{}
\State $chunk \gets threadID * chunkSize$\Comment{Each thread gets a unique chunk}
\State ANSVLinear($chunk$)
\If($chunk$ detects no PSV/NSV)\Comment{ANSVLinear may be wrong}
\State checkMinTree($mintree$)\Comment{Manually check MinTree}
\EndIf
\EndProcedure
\end{algorithmic}
\caption{ANSV Pseudocode}
\label{algorithm:ansv}
\end{figure}

To calculate the needed ANSV values we used the parallel algorithm from Shun and Zhao \cite{shun2013practical}.

The first step is to build a balanced binary tree, where the leaves are elements from SA, and the ancestors are the minima of their children.
Although more efficient algorithms may exist, we decide to take a simpler naive approach and launch a kernel at each level.
Each thread in the kernel calculates for a node the minimum of its two children and stores it into a 1d array.
A 2d array would be easier to index into, but much more difficult to allocate.
(Something about memory locality and indexing into 1d array).

The suffix array is then divided into even divisions.
Each thread uses a stack, in the form of an array, and traditionally solves ANSV for their division.
Because each thread can only see their division, many of the positions will think there is no smaller position, while they may exist in the next or previous division.
To compensate for this, each thread will manually check each position that did not find a smaller value using a search on the previously generated binary tree.

This algorithm will generate the ANSV arrays for each index, although not every index is needed in the final LZ factorization. 
We did experiment with the idea of solving the ANSV problem for a specific index only when needed, but found that in most cases, this was only a little faster or much slower.

\section{LZ Factorization}

The final step is to calculate the LZ factorization.

At first, we attempted to follow the parallel algorithm of Shun and Zhao \cite{shun2013practical}.
In their work, the LPF array is calculated for every position, and then the LZ factorization is solved using a parallel list ranking algorithm.
Like many more recent works, we found that the calculation of the LPF array at every index to be too computationally expensive and wasteful, even on the GPU.
Instead, our work will also employ the lazy LZ factorization, mentioned in \cite{karkkainen2013linear}.
The biggest problem with the lazy LZ factorization was that it is incredibly sequential.
Since it is impossible to know what entries will exist in future points in the LZ factorization, it is a hard problem to parallelize.

\subsection{PLZ}

We propose breaking away from the ideal LZ factorization and using a Parallel LZ factorization (PLZ).
Using PLZ, the string S will be broken into chunks of size c to be worked on individually.
Each thread will be assigned a chunk and traditionally calculate the LZ factorization on it.
The LZ factorization calculated by each thread will be entered unmodified into the final LZ factorization.
The main advantage of this is being able to parallelize the problem, while not incurring too many penalties on the compression ratio.
By doing this, we are also able to limit the amount of work any one thread will do, in an attempt to load balance.
There are several disadvantages that may appear, all of which depend on the original input string.
There is a chance for the PLZ LZ factorization to be larger than the ideal LZ factorization.
There is also an unlikely chance for them to be exactly the same.
We would like to explain the different scenarios to you.

Let's remember that an entry into the final LZ factorization indicates the start of a factor.
The ideal LZ factorization is a sequence of longest previous factors.
When we use PLZ, we are breaking down the LZ factorization into chunks to be worked on in parallel.
At the start of each chunk, we insert a first entry into the LZ factorization.
We then continue calculating the LZ factorization using the traditional sequential algorithm.
At the end of the chunk, we stop the string comparisons and cut off the current factor.
By stopping the string comparisons, we are able to limit the amount of work needed to process a chunk.
This also means that factors are limited in length to the chunk size.

The first scenario occurs when that first entry is in the same position as an ideal LZ factorization factor.
The LZ factorization of that chunk will then be the same as the ideal LZ factorization.
If every first entry is in the same positions as an ideal LZ factorization, the final PLZ LZ factorization will be exactly the same as the ideal LZ factorization.

The next scenarios occur when a first entry is not in the same position as an ideal LZ factorization factor.
This would happen when the chunk splits in the middle of a ideal LZ factorization factor.
In these scenarios, the last factor in the previous chunk will no longer be the longest.
The LPF of that last factor will be shorter than the ideal LZ factorization.
When the LZ factorization is calculated on this chunk, the next factor may or may not start at an ideal LZ factorization factor.
We then begin calculating the LZ factorization of that chunk, starting at that first entry.
If any of the calculated factors begin at the start of the ideal LZ factorization factor, then all factors after will also match the ideal LZ factorization.
It is impossible to know which of these scenarios will occur, since they are all dependent on the input string.

%Let's imagine the LZ factorization of a string to be a stairway with floors.
%Generating the LZ factorization is like constructing said stairway and floors.
%Each floor represents an entry into the LZ factorization.
%Each stair represents a character match in the LPF for the floor beneath it.
%A sequential algorithm would have you start at the ground floor.
%At that point we know where the LPF is located.
%Of course in reality, we know of two possible locations where the LPF might be, but we'll simplify it to one for this analogy.
%When you are on a floor, you, the builder, walk up a stair for each match.
%When the LPF no longer matches, you put a floor to represent a new entry and repeat.
%In the end, the number of stairs should match the number of original characters, and the number of floors represent the length of the ideal LZ factorization.
%
%Now we'll look at what happens when we use PLZ to parallelize the problem.
%First, we remove all the floors.
%Next, to split up the input, we insert ground floors at regular intervals.
%After, we travel the stairway inserting floors, like the sequential algorithm above.
%Finally, we stop when we get to the next ground floor.
%There are several scenarios that can occur.
%The first scenario occurs when an inserted ground floor is inserted where an existing floor used to exist.
%The LZ factorization calculated, starting at this ground floor, is exactly the same as the ideal LZ factorization, up until the next ground floor.
%The extreme of this scenario happens if every inserted ground floor is inserted where a floor used to exist.
%The PLZ factorization is then exactly the same as the ideal LZ factorization.
%
%The next scenarios occurs when the inserted ground floors are inserted in between existing floors from the ideal.
%When this happens, the PLZ LZ factorization is automatically increased by at least one, from the new ground floor.
%The LPF of the previous floor is no longer the longest it can be, since we stop the matching at the end of the chunk.
%This also limits the maximum offset, the longest match in the LZ factorization, to be the chunk size.
%This new ground floor now performs the sequential algorithm as always.
%The next floor that is inserted by the builder may or may not be where a floor existed.
%This is entirely dependent on the input string and is impossible to predict.
%It is very likely though at some point, for the floors to again match the existing floors.
%At that point onward, the PLZ LZ factorization again matches with the ideal.

The next question to be answered is deciding the chunk size c. 
A larger chunk size could reduce the chances for a larger factorization and increase compression ratios.
On the other hand, a smaller chunk size would more evenly distribute the work among the GPU threads, and in turn should increase compression speeds.
This is a trade-off that should be left to the user.
In our implementation, we have the option to define an arbitrary size for the chunk size c or for a number of divisions d of the input string.
Some optimal sizes might be to use divisions that are multiples of the number of multiprocessors.
In any case, it is impossible to predict the compression ratio, and different people will have different priorities.

The resulting PLZ LZ factorization when using PLZ with chunk sizes of 7, 4, and 3 can be seen in Tables~\ref{tab:example7}, \ref{tab:example4}, and \ref{tab:example3}. 
These chunk sizes, from a string size n=14, can result from divisions of d = 2, 4, 5.
Table~\ref{tab:allsolved} can be used as reference with the whole LPF and PrevOcc arrays filled.
First note that the LPF and PrevOcc arrays are not totally filled.
As we are doing a lazy LZ factorization, not all values need to be computed, and this is shown accordingly.
Next, notice that there are breaks within the table.
These indicate chunks for a single thread, or block as we'll see soon, to work on.
The next thing to notice are the bold elements in the LPF array.
A bold element indicates that the value is no longer the LPF at that position and is changed from the reference LPF values.
Recall that when using PLZ, the string matching stops at the end of a chunk.

In Table~\ref{tab:example7} with a chunk size of 7 and a division of 2, we can see that there are no changes in the LZ factorization.
The split occurred at position we = 7, the beginning of a factor in the ideal LZ factorization.
Therefore, we see no changes while being able to solve the problem in parallel.

Tables~\ref{tab:example4} and \ref{tab:example3} begin to show changes from the ideal LZ factorization.
In Table~\ref{tab:example4}, we use a chunk size of 4 and a division of 4.
Notice how the third chunk starts at position 8.
Because the ideal LZ factorization did not have a factor starting at position 8, it can be determined that the PLZ LZ factorization is no longer the same as the LZ factorization.
The LPF at position 8 is 2, so that the next factor starts at position 10.
The ideal LZ factorization had a factor starting at position 10, so we are now back on track.
Any additional factors calculated from this chunk should match that of the ideal LZ factorization.
The resulting PLZ LZ factorization has a length of l=9, 1 more than the ideal LZ factorization length of l=8;

The last example in Table~\ref{tab:example3} shows a chunk size of 3 and a division of 5.
Notice again that chunks 2 and 3, starting at positions 6 and 9 respectively, have a different factor from the ideal LZ factorization.
A key thing to realize from this example is that the length is unchanged.
Both the PLZ LZ factorization and the LZ factorization have a length, l=8.
By using PLZ, we were able to split the work into 5 to be worked on in parallel and compress the string to the same length as the ideal LZ factorization.

\section{More LZ Factorization Optimizations}

One thing that we have yet to cover is how we perform the string match.
The naive operation is to do a character by character match until the prefix no longer matches with the LPF.
One optimization that we have implemented is to instead do a parallel string comparison.
A block of threads can load a group of characters into shared memory.
The BlockLoad primitive from CUB is used to load a number of characters from the LPF and from the prefix into arrays for use in an individual thread in the block.
To simplify, we can imagine a block of 32 threads loading a chunk of 32 characters from the LPF and prefix.
Each thread, responsible for a single index and two characters, then compares the two characters for a match.
A match is assigned the block's id (32), and a mismatch is assigned the thread's id (0-31).
The threads can then cooperatively find the minimum in a block using reduction, with the BlockReduce primitive from CUB for example.
If all 32 characters matched, the value 32 is returned to the first thread in the block, which controls all the logic.
That thread will then set a flag to indicate to the rest of threads to continue with the comparisons.
If there is a mismatch, the index of the mismatch is instead returned to the first thread, which can then stop the comparisons.
Which implementation is faster is solely dependent on the data and the average factor length.
Average factor lengths less than the number of characters loaded and compared in parallel may see faster speeds with just the naive comparisons.
Our implementation will use the parallel string comparison for evaluation.
In doing so, each chunk is worked on by a single block.

Finally because we are cutting off the matching at the end of the chunk, an optimization can be made to reduce the number of searches.
Because the LPF can occur at either the PSV or the NSV, we usually need to check both and pick the longer. 
Reaching the edge of a chunk during the first search allows us to skip the second chunk.

The final LZ factorization, made up of relevant entries from the LPF and PrevOcc arrays, can be gathered by using CUB's DeviceSelect, which allows us to compact the arrays using a flag set at the start of each factor.
This allows our final data transfer to require sending less data.

\chapter{Results}
\label{chap:results}

\section{Experimental Setup}
\subsection{Test Machine}
All measurements were gathered from a single machine with an NVIDIA Tesla K40c and NVIDIA GTX TITAN Black.
The Tesla K40c and GTX TITAN Black are two of NVIDIA's higher end solutions.
The GTX TITAN Black, which we'll now refer to as Black, has a 0.98 GHz GPU clock rate, 3.5 GHz memory clock rate, and 6 GB of memory.
The Tesla K40c, now K40c, has a 0.88 GHz GPU clock rate, 3.0 GHz memory clock rate, and 12 GB of memory.
The Black is faster than the K40c, but has significantly less memory.
Both the CUDA runtime and driver version were 6.0.
The binary was compiled using -O3 optimization and compute capability 2.0.
Timings were recorded using the CUDA events api.

\subsection{Data}

Data from the various tests were gathered from 

\subsection{Validation}

Talk about compare to cpu implementations to validate

\section{Suffix Array}

\begin{figure}[ht!]
\centering
\includegraphics[width=1.0\textwidth]{images/saresult.png}
\caption{Speedup of GPU implementation to fastest CPU implementation}
\label{fig:saresult}
\end{figure}

\begin{table}[h]
\centering
\begin{tabular}{@{}llll@{}}
\toprule
size      & name      & gpu    & ms per input \\ \midrule
53161     & paper1    & 16     & 0.301        \\
111261    & bib       & 21.2   & 0.191        \\
377109    & news      & 36.8   & 0.098        \\
448779    & mj        & 30.9   & 0.069        \\
509519    & hwe        & 36.1   & 0.071        \\
3295751   & hs        & 130.6  & 0.040        \\
4047392   & bible.txt & 141.5  & 0.035        \\
10192446  & dickens   & 353.5  & 0.035        \\
21606400  & samba     & 693    & 0.032        \\
39422105  & howto     & 1339.7 & 0.034        \\
51220480  & mozilla   & 1642.6 & 0.032        \\
69728899  & jdk13c    & 2525.4 & 0.036        \\
104201579 & w3c2      & 3840.8 & 0.037        \\ \bottomrule
\end{tabular}
\caption{Sizes and runtimes of datasets for evaluation of suffix array construction}
\label{tab:sadata}
\end{table}

The evaluation of the suffix array is actually a evaluation  of a reimplementation of the fastest known GPU suffix array construction algorithm (SACA) by Deo and Meely\cite{Deo}.
The benefits and applications of the suffix array has already been detailed in chapter . . .
Deo and Meely's evaluation was done on an AMD Radeon GPU using OpenCL.
Our results on a NVIDIA GPU using CUDA and CUB primitives with ModernGPU's merge path method to mergesort are not expected to be significantly different.
We will be comparing out results to a set of SACA benchmarks found on the libdivsufsort wiki.
That benchmark compares the fastest CPU SACA implementations on a variety of test files.
We will compare our GPU implementaion against the fastest CPU time for each file.
Files were picked to match closely with Deo and Meely's evaluation.
GPU times include parsing the file, transfering the data both ways, and the construction of the suffix array.

Figure \ref{fig:saresult} presents the results comparing the CPU implementations to our GPU implementation.
The first thing to note is that the CPU SACA benchmarks are significantly faster than those used by Deo and Meely.
Our GPU implementation did not see the speedup of 35x that theirs did, but we still found around a 4-5x speedup for most files for Black.
We did not have their implementation or their raw result data to compare against.
Loosely comparing with the charts in their paper though, we find that our GPU implementation is at least on par if not faster.

Another interesting metric is the runtime in microseconds per input symbol seen in \cite{ }.
We found that after a certain point, our GPU implementation was achieving rates of around .03 to .04 ms per input symbol.

Like many other GPU algorithms, we found that smaller files did not see the greater speedups that larger files did.
The likely cause is that smaller files cannot fully saturate the GPU, and the cost of intialization and data transfer could not be hidden by increased computations.
This indicates that the GPU is not the all around solution for faster suffix arrays and the size of the input needs to be considered.

\begin{figure}[ht!]
\centering
\includegraphics[width=1.0\textwidth]{images/saprofile.png}
\caption{Profile of Suffix Array construction on the GPU}
\label{fig:saprofile}
\end{figure}

Figure \ref{fig:saprofile} shows a profile of the SACA of the GPU implementation.
Kernels other than those involved in the merging or sorting take the greatest percentage of time in both examples.
These kernels have the most room for improvement, since they are less likely to deal with primitives and more likely deal with the setup and movement of data.
The CUDA grid and block sizes could have a greater factor in the speeds and further optimization are more likely to see gains here.

aug 24 2008

\section{ANSV}

\begin{figure}[ht!]
\centering
\includegraphics[width=1.0\textwidth]{images/ansvsize.png}
\caption{The effect of chunk size on ANSV runtime on the GPU}
\label{fig:ansvresult}
\end{figure}

As discussed in chapter X, the ANSV algorithm divides the suffix array into chunks for each thread.
These threads will then individually solve the ANSV problem on their chunk and solve any outliers using a preconstructed binary tree.

Figure \ref{fig:ansvresult} shows the impact of changing the chunk size in the ANSV generation.
For our setup we can see a noticeable speedup at a chunk size of 4.
The chunk size of 4 is not a universal speedup for all NVIDIA GPUs.
Altough not presented in this paper, a mobile GPU, NVIDIA GT 650M, found speedups at a much greater chunk size.
Different hardware have different memory latencies and other costs.

Figure \ref{fig:ansvresult} shows a profile of the ANSV generation in the two main steps, the construction of the binary tree and the chunk processing with a chunk size of 4. 
The construction of the min tree was no more than 4 percent of the overall ANSV generation.
Several future optimizations were discussed earlier for the min tree, but seeing as it is only a small percentage of the overall runtime, time is probably better spent somewhere else.
See Amdahl's law \cite{ }.
The bigger chunk of the runtime is in the chunk processing, as expected.

ms per input - ansv

\section{PLZ}
\begin{sidewaysfigure}[ht!]
\centering
\includegraphics[width=1.0\textwidth]{images/chart.png}
\caption{The effects of chunk size on percent increase and runtimes}
\label{fig:lzchart}
\end{sidewaysfigure}

\begin{table}[h]
\centering
\begin{tabular}{llllllllllll}
name             & size  & total  & lz-og & lz-ansv & plz3 & speedup og & speedup ansv & speedup plz & lz      & PLZ     & percent increase \\
10Mrandom.lpf    & 9.5   & 371.1  & 4670  & 3970    & 437  & 12.58      & 10.70        & 1.18        & 1426311 & 1426496 & 0.013         \\
chr22.dna.lpf    & 33.0  & 1385.8 & 22000 & 19400   & 1570 & 15.88      & 14.00        & 1.13        & 2461478 & 2461728 & 0.010         \\
howto.txt.lpf    & 37.6  & 1620.3 & 25500 & 39400   & 1820 & 15.74      & 24.32        & 1.12        & 3063929 & 3064227 & 0.010         \\
jdk13c.lpf       & 66.5  & 2940   & 41400 & 40400   & 2860 & 14.08      & 13.74        & 0.97        & 1209676 & 1210015 & 0.028         \\
wikisamp.xml.lpf & 95.4  & 4304.2 & 61400 & 59900   & 4030 & 14.27      & 13.92        & 0.94        & 2888810 & 2889040 & 0.008         \\
w3c2.lpf         & 99.4  & 4988.5 & 84100 & 63100   & 4420 & 16.86      & 12.65        & 0.89        & 2340638 & 2341016 & 0.016         \\
etext99.lpf      & 100.4 & 5182   & 75200 & 69900   & 4800 & 14.51      & 13.49        & 0.93        & 8306413 & 8306658 & 0.003         \\
sprot34.dat.lpf  & 104.5 & 5210.9 & 72200 & 69000   & 4600 & 13.86      & 13.24        & 0.88        & 6395921 & 6396224 & 0.005         \\
rctail96.lpf     & 109.4 & 5227.3 & 96500 & 70000   & 4770 & 18.46      & 13.39        & 0.91        & 3905843 & 3906149 & 0.008         \\
rfc.lpf          & 111.0 & 5474   & 76600 & 72800   & 4830 & 13.99      & 13.30        & 0.88        & 5656068 & 5656367 & 0.005        
\end{tabular}
\caption{Sizes, runtimes, and speedups of datasets for evaluation of LZ factorization. GPU implementation uses PLZ with 480 divisions}
\label{tab:lzdata}
\end{table}

To directly compare the generation of PLZ to algoritms and implementations generating the ideal LZ factorization would be unfair.
The outputs are totally different, as the PLZ has lost an important property of the ideal LZ factorization, the LPF.
The LPF in the PLZ are no longer the longest, as discussed in our implementation.
What can be done is a relative comparison to previous implementations.
We will present the percent increase of the PLZ from the LZ factorization to help in the evaluation.

The data set and CPU benchmarks will be taken directly from the results in \cite{ }. 
Specifically, we will compare our results to their benchmarks of LZ-OG, the most time efficient algorithm as seen in \cite{ }, LZ-ANSV, the sequential algorithm which computes the LZ factorization without every LPF value using lazy LZ factorization, and their contribution PLZ3, their parallel CPU algorithm using 40 cores with hyper-threading.
LZ-ANSV is the closest sequential algorithm after the ANSV generation, while the ANSV generation algorithm comes from PLZ3.

The first and most important metric to look at is how the PLZ chunk size affects the final LZ factorization size.
If the percent increase is too great, then the usage of PLZ is unacceptable.
What percent increas is too great is a judgement that must be made by each user, as each user will have their own requirements.
To pick the different chunk sizes, we decided to use number of divisions as the parameter, although we could have used the actual block size as mentioned before.
More specifically we used multiples of the number of SMs (15).
To try and get a good spread we used powers of 2 to multiply.

The second most important metric is how the chunk sizes affect the runtimes. 
Since the suffix array construction and the ANSV generation are unrelated, we will keep our focus on the LPF time.
This time assumes the suffix array and ANSV arrays are already present on GPU memory.
It includes the generation of the necessary LPF and prevocc arrays, the isolation of the needed values using CUB's deviceSelect, and the data copy of those values back to the CPU.
Many LZ factorization papers evaluate the runtime of their algorithm starting after the suffix array is in memory.
We will consider that runtime later.

Figure \ref{fig:lzchart} presents the results with these two metrics together.
We show the effect of percent increase and runtime as a function of the number of divisions.
The key difference is A is geometric while B is linear.
We have included averages as a convenience.
First, we notice that the percent increase grows linearly with the number of divisions.
This trend is intuitive as each extra division has a chance to increase the final PLZ length if the division boundary occurs between the ideal LZ factorization.
Next, we notice that the runtimes generally decreases rapidly as we increase the number of divisions.
At some point however, the rapid decreases stops and increasing the divisions further does not have as much effect on the runtime.
As we can see in figure \ref{fig:lzchart}, this occurs at around 480 divisions for both cards.
At 480 divisions the datasets have .01 average percent increase.
For some perspective, a file that compresses to 1 Mb using the ideal LZ factorization would require an additional 105 bytes using PLZ.
At this point, the Black has an average runtime of 303.7 ms, while the K40c has an average runtime of 406.8 ms.
As we increase the number of divisions from 480 to 30720, Black's runtime decreases only 30 percent, while increasing the percent increase over 150 percent to 1.54 percent increase.
Similarly, K40c's runtime decreases only 32 percent.
We will now deem this .01 percent increase acceptable and use these values as we continue the evaluation.

We now take a look at how our GPU implementation compares to the CPU LZ factorization implementations mentioned earlier.
Table \ref{tab:lzdata} tabulates the results and speedups found in our experiments.
Our GPU implementation outperforms LZ-OG and LZ-ANSV on all the data sets.
Black sees speedups between 15-24x compared to LZ-OG and speedups between 13-19x compared to LZ-ANSV.
When compared to the 40 core PLZ3 implementation, Black performs comparatively with speedups of .1-.5x.
The slower, K40c sees speedups of 12-19x and 10-15x against LZ-OG and LZ-ANSV respectively.
K40c performed comparitively with PLZ3 having speedups and slowdowns no more than 0.2x.

\begin{figure}[ht!]
\centering
\includegraphics[width=1.0\textwidth]{images/allprof.png}
\caption{Profile of GPU implementation}
\label{fig:allprof}
\end{figure}

Figure \ref{fig:allprof} shows a profile of the three main sections of our implementation, the SA, the ANSV, and the LZ.
The majority of our implementation, like most LZ factorization implementations, spend most of their time constructing the suffix array.
The suffix array construction takes on average 81 percent of the overall time.

Excluded from the results are highly compressible input.
These files incur an incredible space cost when using PLZ, as each division is likely to add an additional factor to the factorization.
An additional factor with a highly compressable input could and probably will result in very high percent increases.

\chapter{Conclusion}
\label{chap:conclusion}

We have presented an algorithm and implementation to calculate the Lempel-Ziv factorization on the GPU.
Our algorithm requires O(n) work and O(time).
We show the usage of BLZ in the calculation of the LZ factorization.
Although this removed our ability to calculate the ideal LZ factorization that would be calculated in a traditional sequential algorithm, we found that using the BLZ found significant speedups on the GPU, incurring a space cost of at most .01 percent.
Using the BLZ, although not calculating the most space efficient ideal LZ factorization, could work well in a more practical environment.

We have also presented a reimplementaion and reevaluation of the GPU suffix array construction algorithm of Deo and Meely\cite{Deo}.
With the use of GPU libraries and parallel primitives, we were able to replicate their OpenCL results using CUDA on a NVIDIA GPU.
On files greater than 10 mb, we found at least a 3-4x speedup over the fastest CPU implementations.
This suffix array algorithm and implementation have many applications outside of data compression, most notably in bioinformatics.

\chapter{Future Work}

Our implementation, like many other LZ factorizaton implementations, was just a proof of concept to show compression speeds and compression ratio.
Although we do output the correct pairs needed, we could take it further and encode them in a way that decompressers can understand.
In doing so, we could create an actual utility to be used to compress actual data.

One aspect that was not considered in this thesis was the effect of having previous knowledge of the input.
Specifically, what can we do if we know the alphabet of the input is limited.
For instance during the suffix array construction, we do an initial sort of the 2/3 group using a 3 character prefix.
To do this, we need to use three radix sorts.
If we know exactly how many bits represent the largest character or integer in the alphabet, we can specialize the radix sort to only sort on those bits.
If this is not possible, we could also check if three characters could fit into a smaller number of characters and perform a lesser number of radix sorts on them.

Recent work has looked at different ways to solve the ANSV problem.
Work done in \cite{Computing longest previous factor in linear time and applications} and \cite{Simpler and Faster Lempel Ziv Factorization} has explored a technique called peak elimination to solve the ANSV problem.
It is unclear whether this solution would parallelize and fit on the GPU architecture.
They also found success using a single data strucute to hold both the PSVs and NSVs to improve memory locality.

Another very important measurement that we did not consider was the space efficiency of our algorithm.
As inputs, such as DNA sequences, grow larger and larger, it is important to make sure the algorithm is as space efficient as possible, so that the algorithm can scale.
Recent work by \cite{Space Efficient Linear Time Lempel-Ziv Factorization on Constant Size Alphabets} has shown methods to reduce the space needed by LZ factorization algorithms by reusing the space required by auxillary data structures.
It is especially important when working on the GPU, where hardware limits are stricter, memory is more sparse, and the communication overhead to go back and forth from the GPU to the CPU is expensive.

Multiple GPU support is becoming increasingly popular as GPU applications become more mainstream.
Enabling multiple GPU support would allow our implementation to handle larger inputs.
It would be interesting to investigate the added communication overhead and its effects on the overall performance.
Multiple GPU support would also allow us to accompany high performance users, who have machines or clusters of machines with single or multiple GPUs.

A simpler optimization that could be added in future implementations is a more dynamic and adaptable kernel launch parameters.
In our implementation, we left many of the kernel paramters as program launch parameters for exhaustive trial and error.
Other kernel parameters were also optimized specifically to the GPU we used for evaluation.
Our implementation would still work with other GPUs, but different parameters might find faster compression speeds.
One approach is to gather information about the GPU using the CUDA API before launching any kernels.
We can then use that information to generate more sensible grid and block sizes.
We can also use templating features for greater flexibility.
Many CUDA libraries make use of this approach to great success.

NVIDIA CUDA is still a growing framework, as new hardware and new API releases add additional features to ease or enable programmers.
Recent releases have enabled dynamic parallelism and a unified memory.
Dynamic parallelism allows GPU kernels to launch additional kernels from the kernels themselves.
Traditionally, the GPU is used as a coprocessor and kernels must be launched from instructions on the CPU.
Using dynamic parallelism, added overhead from communication between the GPU and CPU can me circumvented.
Dynamic parallelism can also allow for easier or more load balanced parallelism.
For example, during the final LZ factorization calculation, we could launch a new CUDA kernel to do string comparisons instead of having threads in a block working together.
The benefits of unified memory in our current implementation is not clear.
Since most of the data is generated and remains on the GPU, unified memory may only simplify the communication while not adding performance benefits.
It would still be interesting to evaluate if these new features could provide speedups.

As of now, our implementation works only on NVIDIA GPUs through the use of CUDA.
Although GPGPU development is dominated by NVIDIA CUDA on NVIDIA GPUS, the graphics market share includes many other significant vendors, including AMD and Intel.
There are a variety of methods to create an implementation for use with those other vendors.
The first is to rewrite the implementation using OpenCL.
OpenACC, GPU Ocelot
Furthermore, the usage of BLZ could be examined on different platforms, where the cost to perform string comparisons are not as expensive, like a multicore CPU.



% ------------- End main chapters ----------------------

\clearpage
\nocite{*}
%\bibliographystyle{plain}
\bibliography{Bibliography}
%\addcontentsline{toc}{chapter}{Bibliography}

%\begin{appendix}
%%\addcontentsline{toc}{chapter}{\appendixnamelower}
%\include{Appendix}
%\end{appendix}

\end{document}
